\documentclass{article}
\usepackage[noend]{algorithmic}
\usepackage{mathtools}

\title{Proyecto ADA, primer entregable}
\author{Jean Miraval y  Mateo Noel}


\newcommand{\TITLE}[1]{\item[#1]}
\renewcommand{\algorithmiccomment}[1]{$/\!/$ \parbox[t]{4.5cm}{\raggedright #1}}
\newbox\fixbox
\renewcommand{\algorithmicdo}{\setbox\fixbox\hbox{\ {} }\hskip-\wd\fixbox}
\renewcommand{\algorithmicthen}{\setbox\fixbox\hbox{\ {} }\hskip-\wd\fixbox}
\newcommand{\algcost}[2]{\strut\hfill\makebox[1.5cm][l]{#1}\makebox[4cm][l]{#2}}


\begin{document}
	\maketitle
	\section{Introducción}
	Para poder resolver el problema de Min-Matching con el uso de estos algoritmos primero decidimos convertir las cadenas de 0's y 1's en vectores de enteros.\\\\
	Por ejemplo: \\Una cadena A=[00111001], será representada como un vector  A = \{3,1\}. \\\\
	Nota: El código igual recibe como input el tamaño de las cadenas, p, y estas mismas. Solo que las transformará a esta forma de representación.
	
	\section{Resolución} 
	\subsection*{Pregunta 1 (Voraz)} 
	Para el diseño de un algoritmo voraz, se partió desde la intención de elegir las conexiones analizando únicamente la situación actual, es decir en qué bloque de A o B estamos iterando. El objetivo del algoritmo es hallar el matching correcto entre los dos arrays que tenga el peso mínimo. Como se puede ver desde las fórmulas dadas, el peso de un Matching depende de una suma de pesos de diferentes conexiones. Entonces la estrategia voraz analizada deberá de buscar la manera de formar conexiones con el menor peso posible.
	\\\\Estrategia: Armar conexiones minimizando el peso de estas.
	\\\\Observando las fórmulas, podemos concluir que para minimizar el peso de estas, deberíamos procurar que la suma de los pesos de los bloques de B debe de ser mayor que las de A en cada conexión, para así minimizar el peso que resultará de la división. En base a esto, planteamos nuestra estrategia específica, la cual se basa en formar conexiones buscando dentro de lo posible que la suma de los bloques de A sea menos a la suma de los bloques de B.\\\\ Estrategia: Armar conexiones donde la suma de bloques de B sea menor que las de A.\\\\
	A partir de este análisis planteamos un pseudocódigo basado en analizar, a medida que avanzamos por una de las listas, si en todo momento la suma de los pesos de A y B permiten seguir haciendo conexiones de tal manera que cumpla con la estrategia. En caso contrario, se aplicaría un Reset Connection, que terminaría esa conexión e iniciaría una nueva a partir de los siguientes bloques. Se añadió las condiciones de combined y divided para evitar interferir con la condición del problema, donde un bloque de A no puede ser agrupado y dividido a la vez.
	
	Recibe: 
	\begin{itemize}
		\item Vectores A y B
	\end{itemize}
	Devuelve: 
	\begin{itemize}
		\item Peso Min-Matching tipo Float
	\end{itemize}
	\begin{algorithmic}[1]
		\TITLE{\textsc{Min-Matching-Voraz}$(A, B)$}
		\algcost{\textit{cost}}{\textit{times}}
		\STATE Divided=0 \algcost{$.$}{$.$}
		\STATE Combined=0 \algcost{$.$}{$.$}
		\STATE Peso=0 \algcost{$.$}{$.$} 
		\STATE BCurrentw = w(B1) \algcost{$.$}{$.$} 
		\STATE ACurrentw = w(A1) \algcost{$.$}{$.$} 
		\STATE Conectar(1,1) \algcost{$.$}{$.$} 
		\STATE BCurrent = 1 \algcost{$.$}{$.$} 
		\STATE ACurrent = 1\algcost{$.$}{$.$} 
		\FOR{$i=2$ TO max(sizeA,sizeB)}  
		\IF {BCurrent $<$ sizeB and ACurrent $<$ sizeA}
		\IF {ACurrentW $<$ BCurrentW}
		\IF {ACurrent $<$ sizeA-1}
		\IF {Divided or Acurrentw + w(Ai) $>=$ Bcurrentw}
		\STATE \textsc{reset-conection}$(...)$ 
		\ELSE 
		\STATE conectar(i,BCurrent)
		\STATE	ACurrentw = ACurrentw + w(Ai)
		\STATE	ACurrent = i
		\STATE	combined = 1
		\STATE	divided = 0
		\ENDIF
		\ELSE 
		\STATE \textsc{reset-conection}$(...)$ 
		\ENDIF
		\ELSE
		\IF {ACurrent $<$ sizeA-1}
		\IF {if combined or BCurrent $>=$ sizeB-1}
		\STATE \textsc{reset-conection}$(...)$ 
		\ELSE 
		\STATE BCurrent++
		\STATE i--
		\STATE conectar(A[ACurrent], B[BCurrent])
		\STATE divided =1
		\STATE combined=0
		\ENDIF
		\ELSE
		\STATE  \textsc{reset-conection}$(...)$ 
		\ENDIF
		\ELSIF{BCurrent ==sizeB}
		\STATE ACurrent++
		\STATE conectar(A[ACurrent], B[BCurrent])
		\ELSE 
			\STATE BCurrent++;
			\STATE conectar(A[ACurrent], B[BCurrent]);
		\IF {BCurrent == sizeB-1} 
		\STATE break
		\ENDIF
		\ENDIF
		\ENDIF
		\ENDFOR
		\STATE peso += ACurrentw/BCurrentw
		\STATE return peso
	\end{algorithmic}
-----------\\
	\begin{algorithmic}[1] 
	\TITLE{\textsc{Reset-conection}$( ... )$}
	\STATE BCurrent++
	\STATE ACurrent++
	\STATE peso += ACurrentw / BCurrentw 
	\STATE conectar(ACurrent, BCurrent)
	\STATE ACurrentw = w(ACurrent) 
	\STATE BCurrentw = w(BCurrent)
	\end{algorithmic}

-----------\\\\
	\textbf{Análisis de Tiempo}\\
	Como podemos observar en el código, la mayoría de las instrucciones dependen de un costo constante y se repetiría las veces que se ejecute la línea del bucle.\\Este bucle itera el máximo entre el array de bloques de A de tamaño m y  array de bloques de B de tamaño n. Todo el código dependerá del max(m,n)\\
	
	$T(m,n) = O(max\{m,n\})$
	

	\subsection*{Pregunta 2 (Recurrencia)}
	Para plantear el problema con recursividad, y con los conocimientos que se obtuvo al realizar el primer problema, nos dimos cuenta de que el solo añadir un bloque diferente, podría ocasionar en ciertos casos la reestructuración de todas las conexiones realizadas previamente. Por este motivo, no era posible realizar una recurrencia donde el resultado de OPT(i,j) esté relacionado directamente con una respuesta anterior. Por lo tanto, para el diseño de la recurrencia, se tomó en cuenta el uso de la recurrencia únicamente para mapear todas las combinaciones posibles del problema partiendo desde los casos iniciales, para posteriormente comparar cada una de ellas y elegir la más óptima, es decir para el problema. Por razones obvias, esta recurrencia es muy costosa en cuanto a recursos y tiempo.
	\\\\
	Donde m = A.size y n = B.size
	\begin{equation*}
	OPT(i,j) =
	\begin{cases}
	A_i / \sum_{x=j}^{n-1}B_x & i = m-1\\\\
	\sum_{x=i}^{m-1}A_x / B_j & j = n-1\\\\
	min(\\
	min_{w=i}^{m-2}( \sum_{x=i}^{w}A_x/ B_j + OPT(w+1, j+1))\\
	min_{w=j}^{n-2}(A_i/\sum_{x=j}^{w}B_x + OPT(i+1,w+1))\\
	) & \text{caso contrário}
	\end{cases}
	\end{equation*}
	\subsection*{Pregunta 3 (Recursivo) }
	Utilizando la recursividad planteada en la pregunta anterior, se pasó a ejecutar un algoritmo recursivo, donde utilizamos la recurrencia planteada para elegir el caso más óptimo para cada combinación de bloques de longitud m y n posibles con los inputs dados.\\\\
	Recibe:
	\begin{itemize}
		\item a = donde iniciará A
		\item b = donde iniciará B
		\item vector A y B
	\end{itemize}
	Devuelve: 
	\begin{itemize}
		\item Peso Min-Matching tipo Float
	\end{itemize}
	\begin{algorithmic}[1]
		\TITLE{\textsc{Min-Matching-Recursivo}$(A, B, a, b)$}
		\algcost{\textit{cost}}{\textit{times}}
		\IF{a = m-1  \algcost{$.$}{$.$}}  
		\STATE return $A_a / \sum_{x=b}^{n-1}B_x$ \algcost{$.$}{$.$}
		\ENDIF 
		\IF{b = n-1 \algcost{$.$}{$.$}} 
		\STATE return $\sum_{x=a}^{m}A_x / B_b$ \algcost{$.$}{$.$}
		\ENDIF 
		\STATE vector$<$int$>$ posibles \algcost{$.$}{$.$}
		\FOR{$i=a$ TO $m-2$ \algcost{$.$}{$.$}}
		\STATE p = $\sum_{x=a}^{i}A_x/ B_b + $ \\
		$\text{Min-Match-Rec}(A, B, i+1, b+1)$ \algcost{$.$}{$.$}
		\STATE posibles.push\_back(p) \algcost{$.$}{$.$}
		\ENDFOR
		\FOR{$i=b$ TO $n-2$ \algcost{$.$}{$.$}}
		\STATE p = $A_a / \sum_{x=b}^{i}B_x + $ \\
		$\text{Min-Match-Rec}(A, B, a+1, i+1)$ \algcost{$.$}{$.$}
		\STATE posibles.push\_back(p) \algcost{$.$}{$.$}
		\ENDFOR
		\STATE return min(posibles)  \algcost{$.$}{$.$}
	\end{algorithmic}

-----------\\\\
\textbf{Análisis de Tiempo}\\
Para cada \\

$T(m,n) = \Omega(2^{max\{m,n\}})$

	\subsection*{Pregunta 4 (Memoizado)}
	Partiendo desde el código anterior, se le añadirá únicamente el cambio de consulta de memoria del algoritmo, implementada a modo de una matriz. Antes de entrar a cálculos de la recursividad compleja, se consultará si el resultado ya ha sido calculado previamente. Además, en caso se ejecute con una combinación de parámetros que aún no ha sido calculada como solución, se almacenará esta solución a la matriz de memoria. De esta forma, luego de las ejecuciones iniciales, cualquier ejecución del algoritmo con una combinación de parámetros ya calculada, tendrá un tiempo de ejecución constante.
	\\\\
		Recibe:
		\begin{itemize}
			\item a y b, índice de inicio dentro del  subproblema en A y B
			\item Vector A y B
			\item M[m][n] donde se guardarán subproblemas
		\end{itemize}
		Devuelve: 
		\begin{itemize}
		\item Peso Min-Matching tipo Float
		\end{itemize}
	\begin{algorithmic}[1]
		\TITLE{\textsc{Min-Matching-Memoizado}$(A, B, a, b, M)$}
		\algcost{\textit{cost}}{\textit{times}}
		\IF{M[a][b] 	\algcost{$.$}{$.$}}
		\STATE return M[a][b]	\algcost{$.$}{$.$}
		\ENDIF
		\IF{a = m-1  \algcost{$.$}{$.$}}  
		\STATE M[a][b] $= A_a / \sum_{x=b}^{n-1}B_x$ \algcost{$.$}{$.$}
		\STATE return M[a][b] \algcost{$.$}{$.$}
		\ENDIF 
		\IF{b = n-1 \algcost{$.$}{$.$}} 
		\STATE M[a][b] $= \sum_{x=a}^{m}A_x / B_b$ \algcost{$.$}{$.$}
		\STATE return M[a][b] \algcost{$.$}{$.$}
		\ENDIF 
		\STATE vector$<$int$>$ posibles \algcost{$.$}{$.$}
		\FOR{$i=a$ TO $m-2$ \algcost{$.$}{$.$}}
		\STATE p = $\sum_{x=a}^{i}A_x/ B_b + $ \\
		$\text{Min-Match-Mem}(A, B, i+1, b+1,M)$ \algcost{$.$}{$.$}
		\STATE posibles.push\_back(p) \algcost{$.$}{$.$}
		\ENDFOR
		\FOR{$i=b$ TO $n-2$ \algcost{$.$}{$.$}}
		\STATE p = $A_a / \sum_{x=b}^{i}B_x + $ \\
		$\text{Min-Match-Mem}(A, B, a+1, i+1,M)$ \algcost{$.$}{$.$}
		\STATE posibles.push\_back(p) \algcost{$.$}{$.$}
		\ENDFOR
		\STATE M[a][b] = min(posibles)
		\STATE return M[a][b]  \algcost{$.$}{$.$}
	\end{algorithmic}

-----------\\\\
\textbf{Análisis de Tiempo}\\
A medida que se va ejecutando la función vamos guardando valores en una matriz[m][n] y no tendrémos que ejecutar la función cada vez que se encuentre el mismo problema. Se ejecuta solo una iteración del Min-Matching-Memoizado a medida que se llena cada valor de la matriz, 
$T(m,n) <= c*(m*n)$ por ende:\\

$T(m,n) = O(mn)$

\end{document}